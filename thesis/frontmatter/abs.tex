\newpage
\TOCadd{Abstract}

\noindent \textbf{Supervisory Committee}
\tpbreak
\panel

\begin{center}
\textbf{ABSTRACT}
\end{center}

Advances in computer technology have enabled the collection,
digitization and automated processing of huge archives of bioacoustic
sound.  Many of the tools previously used in bioacoustics work well
with small to medium-sized audio collections, but are challenged when
processing large collections of tens of terabytes to petabyte size.
In this thesis, a system is presented that assists researchers to
listen to, view, annotate and run advanced audio feature extraction
and machine learning algorithms on these audio recordings.  This
system is designed to scale to petabyte size.  In addition, this
system allows citizen scientists to participate in the process of
annotating these large archives using a casual game metaphor.  In this
thesis, the use of this system to annotate a large audio archive
called the Orchive will be evaluated.  The Orchive contains over
20,000 hours of orca vocalizations collected over the course of 30
years, and represents one of the largest continuous collections of
bioacoustic recordings in the world.  The effectiveness of our
semi-automatic approach for deriving knowledge from these recordings
will be evaluated and results showing the utility of this system will
be shown.
