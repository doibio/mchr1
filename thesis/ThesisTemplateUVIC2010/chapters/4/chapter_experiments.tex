\startchapter{Experiments}
\label{chapter:Exp}

Assuming you have some experimental results to support your claims this is where all the data is reported. There are a few issues you should consider before dumping a lot of stuff here, or it will lose its effectiveness.

First of all you must describe precisely the experimental setup and the benchmarks you used. In any scientific discipline an experimental result is only good if it is reproducible. To be reproducible then somebody else must have sufficient details of the setup to be able to obtain the same data. Thus the first section in this chapter is a super precise history of the decisions made towards experimentation, including mentions of the paths which became infeasible. The setup must be valid and thus your description of it must prove that it is indeed sound. At times, terrifying times, when writing this section, both supervisor and student realize belatedly that something is missing and more work needs to be done!

The second portion of this chapter is dedicated the the actual results. At least two issues arise here:
\begin{enumerate}
\item {Should all the data be reported here or should some be placed in the Appendix?}
\item {Should this be an exposition of the raw facts and data or should it include its analysis and evaluation?}
\end{enumerate}

There are no definite answers here, but I follow a few rules.

\textit{Should all the data be reported here or should some be placed in the Appendix?}
    \begin{itemize}
    \item{If there is a large number of tables of data, it might be better to present here only a handful of the most significant ("best") results, leaving all the rest of the data in the Appendix with proper linkages, as it would make the chapter so much more easily readable (not to mention limiting the struggle with a word processor for the proper placement of tables and text).}
    \item{Use an example throughout, call it a "case study" to make it sound better, so that all the data and results are somehow linked in their logic, and even better if this is one of the examples you used in Chapter 2 to describe the original problem.}
    \item{Highlight in some manner the important new data, for example the column of your execution speed where all the numbers are much smaller. Make the results highly easy to read!}
    \item{It is normally expected that data should be presented only in one form and not duplicated, that is, you are not supposed to include both a table of raw numbers and also its graphical representation from some wonderful Excel wizard. I tend to disagree. I would not wish to see every results repeated in this manner, but some crucial ones need to be seen in different manners, even with the same information content, in order to show their impact. One good trick is to place the more boring tables in the Appendix and use wonderful graphs in this chapter.}
    \item{This is the one chapter where I would splurge and use colour printing where necessary, as it makes an \textit{enourmous} difference.}
    \end{itemize}

\textit{Should this be an exposition of the raw facts and data or should it include its analysis and evaluation?}
     \begin{itemize}
    \item{Is the evaluation of the data really obvious? For example you have 10 tables to show that your chemical process is faster in development and gives purer material - you may simply need to highlight one column in each table and state the obvious.}
    \item{Most results are not that obvious even if they appear so. Moreover this is where you are comparing your \textit{new} results to data from other people. I usually describe other people's work at this point and make comparisons. That is why I prefer to talk about the analysis and evaluation of the results in a separate chapter.}
    \item{There is absolutely no clear structure here which is best.}
    \end{itemize}
